\cleardoublepage
\thispagestyle{empty}
\begin{center}
\begin{LARGE}
\textbf{Resumen}
\end{LARGE}
\end{center}
\begin{quotation}


Actualmente, la inseguridad y el robo de motocicletas representan una preocupación creciente en Paraguay, especialmente para las personas que dependen de este medio de transporte para su vida cotidiana. En respuesta a esta problemática, se desarrolló un sistema de seguridad y rastreo para motocicletas utilizando tecnologías de \textit{IoT}, integrando componentes como \textit{RFID} para la autenticación de usuarios, \textit{LoRaWAN} para la transmisión de datos y la red \textit{Helium} como infraestructura de conectividad. Este sistema permite que solo usuarios autorizados puedan encender y utilizar la motocicleta, y facilita el monitoreo de su ubicación en tiempo real para favorecer su recuperación en caso de robo.

El objetivo del sistema es permitir la autenticación segura del usuario mediante \textit{RFID} y habilitar el arranque del motor solo para usuarios autorizados, además de rastrear en tiempo real la ubicación de la motocicleta y enviar alertas en caso de intento de robo. La metodología incluyó el diseño de un prototipo de hardware, el desarrollo de \textit{firmware} para la interacción entre módulos de autenticación, geolocalización y transmisión de datos, y pruebas en entornos controlados y en campo.  Las pruebas realizadas, tanto en laboratorio como en campo, demostraron una tasa de éxito del 95.56\%, destacando la estabilidad y efectividad del sistema incluso en movimiento. Comparado con métodos de seguridad tradicionales, esta solución combina autenticación continua y monitoreo en tiempo real, mostrando su viabilidad como alternativa para la seguridad vehicular.


\vspace*{0.5cm}

\noindent \textbf{Descriptores: Seguridad vehicular, \textit{IoT}, \textit{RFID}, \textit{LoRaWAN}, \textit{Helium}, Geolocalización} .
\end{quotation}
