\cleardoublepage
\thispagestyle{empty}
\begin{center}
\begin{LARGE}
\textbf{Abstract}
\end{LARGE}
\end{center}

\begin{quotation}

Currently, insecurity and motorcycle theft represent a growing concern in Paraguay, especially for people who rely on this means of transport for their daily lives. In response to this issue, a security and tracking system for motorcycles was developed using IoT technologies, integrating components such as RFID for user authentication, LoRaWAN for data transmission, and the Helium network as connectivity infrastructure. This system allows only authorized users to start and operate the motorcycle and facilitates real-time location monitoring to aid in recovery in case of theft.

The objective of the system is to enable secure user authentication through RFID and to allow motor start-up only for authorized users, as well as to track the motorcycle's location in real-time and send alerts in case of attempted theft. The methodology included the design of a hardware prototype, firmware development for interaction between authentication, geolocation, and data transmission modules, and testing in controlled and field environments. Tests conducted in both laboratory and field environments demonstrated a 95.56\% success rate, highlighting the system's stability and effectiveness, even during movement. Compared to traditional security methods, this solution combines continuous authentication and real-time monitoring, proving its feasibility as an alternative for vehicle security.

\vspace*{0.5cm}

\noindent \textbf{Keywords: Vehicle security, IoT, RFID, LoRaWAN, Helium, Geolocation}.
\end{quotation}
