\fancyhead{}
\fancyfoot{}
\pagestyle{plain}

\lhead{Tecnologías utilizadas}

\chapter{Tecnologías utilizadas}

En este capítulo se presentan las herramientas utilizadas para el desarrollo de este trabajo. En primer lugar, para asegurar que las herramientas seleccionadas sean adecuadas, se realiza una comparación entre alternativas considerando los factores y características más importantes dentro de las actividades.

\section{Componentes de \textit{Hardware}}

Los componentes de hardware abarcan los dispositivos físicos necesarios para la ejecución del sistema, incluyendo unidades de procesamiento gráfico especializadas que se encargan de realizar las tareas de entrenamiento y procesamiento de modelos de inteligencia artificial requeridas.

\subsection{Tarjeta NVIDIA® GeForce RTX™ 5090}

La \textit{NVIDIA® GeForce RTX™ 5090} (véase Figura \ref{fig:rtx5090}) es una unidad de procesamiento gráfico (\textit{\acrshort{gpu-acronym}}) de alta gama diseñada específicamente para aplicaciones de inteligencia artificial, aprendizaje automático y computación de alto rendimiento. Esta tarjeta representa el estado del arte en tecnología de procesamiento paralelo, ofreciendo capacidades excepcionales para el entrenamiento de modelos de \textit{deep learning} y procesamiento de grandes volúmenes de datos.

La \textit{RTX™ 5090} incorpora la arquitectura más avanzada de \textit{NVIDIA}, con miles de núcleos \textit{CUDA} especializados que permiten la ejecución paralela de operaciones matemáticas complejas. Esta característica es fundamental para el entrenamiento de redes neuronales profundas, donde se requiere procesar matrices de gran tamaño y realizar millones de operaciones por segundo. Además, cuenta con núcleos \textit{Tensor} de tercera generación que aceleran específicamente las operaciones de inteligencia artificial, reduciendo significativamente los tiempos de entrenamiento \cite{NVIDIA2024RTX5090}.

\begin{figure}[H]
\leavevmode
\begin{minipage}{\textwidth}
\begin{center}
\includegraphics[scale=0.3]{./capitulo_03/figures/HW/RTX5090.jpg}
\caption{Tarjeta gráfica NVIDIA® GeForce RTX™ 5090\label{fig:rtx5090}}
\end{center}
\end{minipage}
\end{figure}
w
Una de las ventajas más destacadas de esta \textit{\acrshort{gpu-acronym}} es su amplia memoria \textit{VRAM} de alta velocidad, que permite cargar y procesar \textit{datasets} de gran tamaño sin limitaciones de memoria. Esto resulta especialmente importante cuando se trabaja con modelos de visión computacional que requieren procesar imágenes de alta resolución o grandes cantidades de datos simultáneamente.

Para este proyecto, la obtención de la tarjeta \textit{NVIDIA® GeForce RTX™ 5090} se realizó a través de la plataforma \textit{NiceGPU}, un servicio en línea que proporciona acceso gratuito a recursos de computación \textit{\acrshort{gpu-acronym}} de alto rendimiento para fines académicos y de investigación. Esta plataforma permite a estudiantes e investigadores acceder a hardware especializado sin incurrir en los altos costos asociados con la adquisición de equipos de gama profesional.

La selección de esta \textit{\acrshort{gpu-acronym}} se fundamentó en un análisis comparativo de diferentes opciones disponibles en el mercado. Se consideraron criterios técnicos específicos para determinar la opción más adecuada para las demandas computacionales del proyecto.

Los criterios de evaluación establecidos fueron:

\begin{itemize}
    \item \textbf{Rendimiento computacional:} Capacidad para manejar operaciones de matriz y álgebra lineal de forma eficiente.
    \item \textbf{Memoria disponible:} Cantidad de \textit{VRAM} necesaria para cargar modelos complejos y \textit{datasets} grandes.
    \item \textbf{Soporte para \textit{frameworks}:} Compatibilidad nativa con \textit{TensorFlow}, \textit{PyTorch} y otras herramientas de \textit{ML}.
    \item \textbf{Eficiencia energética:} Relación optimizada entre rendimiento y consumo energético.
    \item \textbf{Disponibilidad:} Accesibilidad a través de plataformas de computación en la nube.
    \item \textbf{Costo:} Consideración de los gastos asociados al uso del hardware.
\end{itemize}

Para reflejar el desempeño relativo de cada dispositivo, se empleó una escala del 1 al 3:

\begin{itemize}
    \item \textbf{3:} Indica el mejor rendimiento o la opción más favorable en ese criterio.
    \item \textbf{2:} Representa un rendimiento intermedio, adecuado pero no el mejor.
    \item \textbf{1:} Refleja un rendimiento limitado en comparación con los otros dispositivos.
\end{itemize}

A continuación, se presenta la tabla \ref{fig:Elec_GPU} comparativa que resume las características técnicas más relevantes de cada dispositivo, facilitando así la selección de la opción más adecuada para este proyecto.

\begin{table}[H]
\centering
\renewcommand{\arraystretch}{1.2}
\caption{Tabla comparativa entre Tarjetas Gráficas}
\label{fig:Elec_GPU}
\begin{tabular}{|p{4.5cm}|p{2.2cm}|p{2.8cm}|p{2.5cm}|}
\hline
\textbf{Criterio/GPU}         & \textbf{RTX™ 4080} & \textbf{RTX™ 5090} & \textbf{RTX™ 4090} \\ \hline
Rendimiento computacional     & 2                  & 3                  & 2                  \\ \hline
Memoria disponible           & 2                  & 3                  & 3                  \\ \hline
Soporte para frameworks      & 3                  & 3                  & 3                  \\ \hline
Eficiencia energética        & 3                  & 2                  & 1                  \\ \hline
Disponibilidad               & 2                  & 3                  & 2                  \\ \hline
Costo                        & 2                  & 3                  & 1                  \\ \hline
\textbf{Total}               & \textbf{14}        & \textbf{17}        & \textbf{12}        \\ \hline
\end{tabular}
\end{table}

De la tabla comparativa se determina que la \textit{NVIDIA® GeForce RTX™ 5090} es el dispositivo más adecuado para la investigación. Este dispositivo fue seleccionado para las pruebas debido a su superior rendimiento computacional y disponibilidad gratuita a través de \textit{NiceGPU}, lo que lo convierte en una opción ideal para el entrenamiento de modelos de inteligencia artificial complejos.

\section{Componentes de \textit{Software}}

Los componentes de software proporcionan las herramientas y algoritmos necesarios para diseñar, entrenar y evaluar los modelos de inteligencia artificial. Además, facilitan la preparación y gestión de \textit{datasets}, permitiendo un flujo de trabajo eficiente desde la recolección de datos hasta la implementación del modelo final.

\subsection{Algoritmos de Entrenamiento}

Para el desarrollo del modelo de inteligencia artificial se seleccionaron tres arquitecturas de redes neuronales convolucionales reconocidas por su eficacia en tareas de visión computacional. La selección se basó en un análisis comparativo considerando factores como precisión, eficiencia computacional y capacidad de generalización.

\subsubsection{ResNet50}

\textit{ResNet50} (\textit{Residual Network} de 50 capas) es una arquitectura de red neuronal convolucional profunda que introduce el concepto de conexiones residuales para resolver el problema de degradación en redes muy profundas. Esta arquitectura permite el entrenamiento efectivo de redes con mayor número de capas sin sufrir el problema del gradiente que desaparece.

Las conexiones residuales permiten que la información fluya directamente a través de la red mediante atajos, facilitando el entrenamiento de arquitecturas más profundas y complejas. \textit{ResNet50} ha demostrado excelente rendimiento en tareas de clasificación de imágenes y transferencia de aprendizaje, convirtiéndose en una opción robusta para aplicaciones de visión computacional \cite{He2016Deep}.

La arquitectura \textit{ResNet50} se caracteriza por:

\begin{itemize}
    \item \textbf{Estabilidad en el entrenamiento:} Las conexiones residuales facilitan la convergencia del modelo durante el proceso de entrenamiento.
    \item \textbf{Transferencia de aprendizaje:} Disponibilidad de pesos pre-entrenados en \textit{ImageNet} que aceleran el desarrollo.
    \item \textbf{Rendimiento comprobado:} Resultados consistentes y reproducibles en \textit{benchmarks} académicos.
    \item \textbf{Eficiencia computacional:} Balance adecuado entre precisión y recursos computacionales necesarios.
\end{itemize}

\subsubsection{EfficientNetB3}

\textit{EfficientNetB3} forma parte de la familia \textit{EfficientNet}, que se caracteriza por optimizar simultáneamente la profundidad, anchura y resolución de la red neuronal mediante un método de escalado compuesto. Esta arquitectura logra un balance superior entre precisión y eficiencia computacional.

\textit{EfficientNetB3} utiliza bloques de convolución eficientes y técnicas de escalado que permiten obtener mejor rendimiento con menor cantidad de parámetros comparado con arquitecturas tradicionales. Su diseño optimizado resulta especialmente valioso cuando se trabaja con recursos computacionales limitados o se requiere implementación en dispositivos con restricciones de memoria \cite{Tan2019EfficientNet}.

Las ventajas principales de \textit{EfficientNetB3} incluyen:

\begin{itemize}
    \item \textbf{Eficiencia de parámetros:} Menor número de parámetros para rendimiento equivalente o superior.
    \item \textbf{Escalado balanceado:} Optimización simultánea de múltiples dimensiones de la red neuronal.
    \item \textbf{Velocidad de inferencia:} Tiempos de procesamiento reducidos sin comprometer la precisión.
    \item \textbf{Versatilidad:} Adaptabilidad a diferentes tamaños de \textit{dataset} y dominios de aplicación.
\end{itemize}

\subsubsection{Vision Transformer (ViT)}

\textit{Vision Transformer} (\textit{ViT}) representa un paradigma diferente en el procesamiento de imágenes, aplicando la arquitectura \textit{Transformer} (originalmente diseñada para procesamiento de lenguaje natural) al dominio de la visión computacional. Esta aproximación divide las imágenes en \textit{patches} que son tratados como \textit{tokens} de secuencia.

\textit{ViT} ha demostrado capacidades superiores cuando se dispone de grandes cantidades de datos de entrenamiento, ofreciendo una alternativa prometedora a las redes convolucionales tradicionales. Su mecanismo de atención permite capturar relaciones globales en la imagen de manera más directa que las convoluciones \cite{Dosovitskiy2020Image}.

Los beneficios de \textit{ViT} incluyen:

\begin{itemize}
    \item \textbf{Atención global:} Capacidad para modelar relaciones a larga distancia en imágenes de forma directa.
    \item \textbf{Escalabilidad:} Rendimiento mejorado con \textit{datasets} más grandes y diversos.
    \item \textbf{Flexibilidad:} Menor \textit{inductive bias} comparado con \textit{CNNs} tradicionales.
    \item \textbf{Innovación arquitectural:} Representación del estado del arte en visión computacional moderna.
\end{itemize}

Con el fin de seleccionar los algoritmos más adecuados, se establecieron criterios de evaluación que permiten analizar y comparar las opciones disponibles:

\begin{itemize}
    \item \textbf{Precisión:} Capacidad del modelo para realizar clasificaciones correctas en datos de prueba.
    \item \textbf{Eficiencia computacional:} Relación entre rendimiento y recursos computacionales requeridos.
    \item \textbf{Velocidad de entrenamiento:} Tiempo necesario para convergencia del modelo.
    \item \textbf{Generalización:} Capacidad para mantener rendimiento en datos no vistos durante el entrenamiento.
    \item \textbf{Transferencia de aprendizaje:} Disponibilidad y efectividad de modelos pre-entrenados.
    \item \textbf{Documentación y soporte:} Calidad de recursos disponibles y comunidad de desarrolladores.
\end{itemize}

La tabla \ref{fig:Elec_Algoritmos} presenta la comparación cuantitativa de los algoritmos seleccionados:

\begin{table}[H]
\centering
\renewcommand{\arraystretch}{1.2}
\caption{Tabla comparativa entre Algoritmos de Entrenamiento}
\label{fig:Elec_Algoritmos}
\begin{tabular}{|p{4.5cm}|p{2.2cm}|p{2.5cm}|p{2.0cm}|}
\hline
\textbf{Criterio/Algoritmo}       & \textbf{ResNet50} & \textbf{EfficientNetB3} & \textbf{ViT} \\ \hline
Precisión                        & 3                 & 3                       & 3            \\ \hline
Eficiencia computacional         & 2                 & 3                       & 2            \\ \hline
Velocidad de entrenamiento       & 3                 & 2                       & 1            \\ \hline
Generalización                   & 3                 & 3                       & 2            \\ \hline
Transferencia de aprendizaje     & 3                 & 3                       & 2            \\ \hline
Documentación y soporte          & 3                 & 2                       & 3            \\ \hline
\textbf{Total}                   & \textbf{17}       & \textbf{16}             & \textbf{13}  \\ \hline
\end{tabular}
\end{table}

De acuerdo con los resultados obtenidos en la tabla comparativa, los tres algoritmos presentan características complementarias que justifican su selección conjunta. \textit{ResNet50} destaca por su estabilidad y soporte, \textit{EfficientNetB3} por su eficiencia, y \textit{ViT} por su innovación arquitectural, proporcionando un conjunto diverso de aproximaciones para el entrenamiento del modelo.

\subsection{Plataforma Roboflow}

\textit{Roboflow} (ver Figura \ref{fig:roboflow}) es una plataforma en línea especializada en la gestión, preparación y \textit{augmentación} de \textit{datasets} para proyectos de visión computacional. Esta herramienta facilita todo el flujo de trabajo desde la recolección de imágenes hasta la preparación de datos listos para el entrenamiento de modelos de inteligencia artificial.

\begin{figure}[H]
\leavevmode
\begin{minipage}{\textwidth}
\begin{center}
\includegraphics[scale=0.4]{./capitulo_03/figures/SW/roboflow.png}
\caption{Interfaz de la plataforma Roboflow\label{fig:roboflow}}
\end{center}
\end{minipage}
\end{figure}

La plataforma ofrece funcionalidades esenciales para el procesamiento de \textit{datasets}:

\textbf{Gestión de Datos:} \textit{Roboflow} permite cargar, organizar y etiquetar imágenes de manera eficiente. Su interfaz intuitiva facilita la anotación de objetos en imágenes, generando automáticamente los archivos de etiquetas en formatos compatibles con los principales \textit{frameworks} de \textit{machine learning}.

\textbf{\textit{Augmentación} Automática:} Una de las características más valiosas es su capacidad para aplicar técnicas de \textit{augmentación} de datos de forma automática. Esto incluye rotaciones, cambios de iluminación, recortes, y otras transformaciones que incrementan la diversidad del \textit{dataset} y mejoran la generalización del modelo.

\textbf{Preprocesamiento:} La plataforma automatiza tareas de preprocesamiento como redimensionamiento, normalización y conversión de formatos, asegurando que los datos estén optimizados para el entrenamiento.

\textbf{Control de Calidad:} Herramientas integradas para detectar imágenes duplicadas, etiquetas inconsistentes y otros problemas comunes que pueden afectar la calidad del \textit{dataset}.

\textbf{Exportación Flexible:} Capacidad para exportar \textit{datasets} en múltiples formatos compatibles con \textit{TensorFlow}, \textit{PyTorch}, \textit{YOLOv5}, y otros \textit{frameworks} populares \cite{Roboflow2023}.

Para la selección de la plataforma de gestión de \textit{datasets}, se evaluaron diferentes opciones considerando criterios específicos para el desarrollo del proyecto. Se compararon \textit{Roboflow}, \textit{Labelbox} y \textit{V7} bajo los siguientes parámetros:

\begin{itemize}
    \item \textbf{Facilidad de uso:} Intuitividad de la interfaz y curva de aprendizaje requerida.
    \item \textbf{Automatización:} Capacidades de automatización en el procesamiento de datos.
    \item \textbf{Calidad:} Herramientas disponibles para asegurar la consistencia del \textit{dataset}.
    \item \textbf{Compatibilidad:} Soporte nativo para múltiples formatos y \textit{frameworks}.
    \item \textbf{Costo:} Disponibilidad de tier gratuito adecuado para proyectos académicos.
    \item \textbf{Documentación:} Calidad de recursos y soporte de la comunidad.
\end{itemize}

La tabla \ref{fig:Elec_Dataset} presenta la comparación entre las plataformas evaluadas:

\begin{table}[H]
\centering
\renewcommand{\arraystretch}{1.2}
\caption{Tabla comparativa entre Plataformas de Gestión de Datasets}
\label{fig:Elec_Dataset}
\begin{tabular}{|p{4.5cm}|p{2.2cm}|p{2.2cm}|p{2.2cm}|}
\hline
\textbf{Criterio/Plataforma}     & \textbf{Roboflow} & \textbf{Labelbox} & \textbf{V7} \\ \hline
Facilidad de uso                & 3                 & 2                 & 2           \\ \hline
Automatización                  & 3                 & 2                 & 3           \\ \hline
Calidad                         & 3                 & 3                 & 2           \\ \hline
Compatibilidad                  & 3                 & 2                 & 2           \\ \hline
Costo                           & 3                 & 1                 & 2           \\ \hline
Documentación                   & 3                 & 3                 & 2           \\ \hline
\textbf{Total}                  & \textbf{18}       & \textbf{13}       & \textbf{13} \\ \hline
\end{tabular}
\end{table}

De acuerdo con los resultados obtenidos en la tabla comparativa, \textit{Roboflow} ha sido seleccionado como la plataforma más adecuada para la gestión de \textit{datasets} en este proyecto. Su combinación de facilidad de uso, capacidades de automatización avanzadas, y disponibilidad de un tier gratuito robusto lo posicionan como la opción óptima para el desarrollo académico.

Estas tecnologías, tanto de \textit{hardware} como \textit{software}, fueron seleccionadas considerando su complementariedad y capacidad para soportar los objetivos específicos del proyecto. La combinación de la potencia computacional de la \textit{NVIDIA® GeForce RTX™ 5090}, la diversidad algorítmica de \textit{ResNet50}, \textit{EfficientNetB3} y \textit{ViT}, junto con las capacidades de gestión de datos de \textit{Roboflow}, proporcionan un ecosistema tecnológico robusto y eficiente para el desarrollo del sistema de inteligencia artificial propuesto.