\fancyhead{}
\fancyfoot{}
\pagestyle{plain}

\lhead{Conclusión}

\chapter{Conclusión}

En este capítulo se interpretan y sintetizan los resultados expuestos previamente, además de evaluar la validez de la hipótesis, el logro de los objetivos y realizar recomendaciones para futuros estudios.\\ 

\section{Discusión}
Se ha desarrollado un sistema de seguridad para motocicletas basado en tecnologías de autenticación \textit{RFID} y transmisión de datos mediante \textit{LoRaWAN}. Este sistema, tras varias etapas de desarrollo e investigación en el ámbito \textit{IoT}, ha sido capaz de autenticar usuarios y rastrear la ubicación de la motocicleta en tiempo real.

Durante el proceso de desarrollo, se lograron integrar exitosamente los módulos de autenticación, comunicación y control, permitiendo una funcionalidad de seguridad que restringe el uso de la motocicleta solo a usuarios autenticados y proporciona un monitoreo continuo en caso de situaciones de riesgo.

Además, se tuvieron en cuenta aspectos como la alimentación del sistema en entornos de prueba y la integración de sensores que permitieran tanto la autenticación mediante \textit{RFID} como la recolección de datos de ubicación a través del módulo \textit{GNSS}. Esto aseguró que el prototipo respondiera adecuadamente en condiciones reales de uso en motocicletas.

Las pruebas realizadas arrojaron resultados favorables, evidenciando la efectividad del sistema en el control de acceso y el rastreo de ubicación. Comparado con métodos de seguridad vehicular que solo emplean bloqueos mecánicos o electrónicos, este sistema proporciona una capa adicional de seguridad, permitiendo la autenticación continua del usuario y la desconexión automática en caso de que el usuario se aleje del vehículo. Esta solución, probada en entornos urbanos simulados, ha mostrado un grado de efectividad del 95.56\% en las situaciones de prueba.

En comparación con otros estudios y tecnologías de seguridad, como los sistemas de rastreo \textit{GPS} convencionales o las aplicaciones de bloqueo remoto, nuestro sistema destaca por su integración de una red de baja potencia (\textit{LoRaWAN}) que permite un monitoreo constante sin afectar significativamente la autonomía de la batería del vehículo. De este modo, el sistema propuesto ofrece un enfoque novedoso y efectivo en términos de seguridad y rastreo.

\section{Análisis de la hipótesis}

A partir de los resultados obtenidos, la hipótesis planteada inicialmente, que sugiere que la integración de tecnologías \textit{RFID} y \textit{LoRaWAN} para la seguridad y rastreo de motocicletas puede ofrecer una alternativa eficaz a los sistemas actuales, ha sido validada. Los datos recopilados durante las pruebas en entornos controlados y en campo confirman que el sistema desarrollado proporciona una capa adicional de seguridad y monitoreo en tiempo real.

\section{Principales logros alcanzados}
En consecuencia al cumplimiento de los objetivos también presentados en el primer capítulo, han sido alcanzados los siguientes logros:

\begin{itemize}
    \item Se definió una estructura para el sistema de seguridad, integrando tecnologías de comunicación y control.
    \item Se desarrolló un \textit{firmware} eficiente para el microcontrolador que permite la interacción fluida entre los dispositivos de seguridad y comunicación.
    \item Se adaptó y aprovechó el uso de \textit{ThingsBoard} como interfaz de usuario, permitiendo la visualización clara de alertas y datos en tiempo real.
    \item Se logró ensamblar un prototipo de \textit{hardware} que integra sensores, actuadores y un sistema de alimentación adecuado, garantizando su funcionalidad en el entorno de prueba.
    \item Se logró diseñar y fabricar un encapsulado \textit{3D} personalizado que asegura la protección eficaz de los componentes del sistema.
    \item Se llevaron a cabo pruebas de laboratorio y de campo que confirmaron el rendimiento esperado del sistema, validando su efectividad en escenarios reales y obteniendo datos clave para futuros desarrollos.
\end{itemize}

\section{Sugerencias para futuras investigaciones}
Con base en este trabajo, se sugieren las siguientes áreas de mejora y futuras investigaciones:
\begin{itemize}
    \item Optimizar el sistema de energía mediante algoritmos de bajo consumo para prolongar la vida útil de la batería de respaldo en periodos de inactividad.
    \item Desarrollar una aplicación móvil que permita monitorear el estado del vehículo en tiempo real y recibir alertas inmediatas.
    \item Evaluar la cobertura de la red \textit{Helium} en distintos entornos geográficos y analizar su rendimiento en áreas rurales y urbanas para validar la robustez del sistema.
\end{itemize}

