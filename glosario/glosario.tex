% \fancyhead{}
% \fancyfoot{}


% % \lhead{Glosario}
% % \begin{LARGE}
% % \textbf{Glosario}
% % \end{LARGE}

% % \lhead{Glosario.}
% % %\rhead{\today}
% % %\rfoot{\thepage}

% % \chapter{Glosario}

\newglossaryentry{nwkskey-glossary}{
    name={Network Session Key},
    description={Clave de cifrado utilizada en LoRaWAN para proteger la comunicación entre el dispositivo final y el servidor de red, garantizando la autenticidad y la integridad de los datos transmitidos.}
}
\newacronym{nwkskey-acronym}{NwkSKey}{Clave de Sesión de Red}

\newacronym{appskey-acronym}{AppSKey}{Clave de Sesión de Aplicación}
\newglossaryentry{appskey-glossary}{
    name={Application Session Key},
    description={Clave de cifrado utilizada en LoRaWAN para garantizar la confidencialidad de los datos entre el dispositivo final y el servidor de aplicación.}
}

\newacronym{appkey-acronym}{AppKey}{Application Key (Clave de Aplicación)}





\newacronym{adr-acronym}{ADR}{Tasa de Datos Adaptativa}
\newglossaryentry{adr-glossary}{
    name={Adaptive Data Rate},
    description={Mecanismo en LoRaWAN que ajusta automáticamente la velocidad de transmisión de datos y la potencia según las condiciones de la señal.}
}



\newacronym{wsn-acronym}{WSN}{redes de sensores inalámbricos}
\newglossaryentry{wsn-glossary}{
    name={wireless sensor networks},
    description={Mecanismo en LoRaWAN que ajusta automáticamente la velocidad de transmisión de datos y la potencia según las condiciones de la señal.}
}

\newacronym{ism-acronym}{ISM}{Industrial, Scientific, and Medical (Banda Industrial, Científica y Médica)}

\newglossaryentry{ism-glossary}{
    name={Industrial, Scientific, and Medical},
    description={Conjunto de bandas de frecuencia de radio designadas a nivel mundial para el uso de equipos industriales, científicos y médicos, así como para tecnologías inalámbricas como Wi-Fi, Bluetooth y LoRa.}
}



\newacronym{lpwan-acronym}{LPWAN}{ (Red de Área Amplia de Bajo Consumo)}


\newglossaryentry{aes}{name={Advanced Encryption Standard},description={Estándar de cifrado avanzado utilizado para asegurar la transmisión de datos en redes como LoRaWAN.}}

\newacronym{aes-acronym}{AES}{Estándar de Cifrado Avanzado}

\newglossaryentry{rtos}{name={FreeRTOS},description={Sistema operativo en tiempo real para microcontroladores y pequeños microprocesadores.}}

\newglossaryentry{api}{name={Application Programming Interface},description={Conjunto de herramientas y definiciones que permiten la interacción entre aplicaciones de software.}}
% \newglossaryentry{arduino}{name={Arduino},description={Plataforma de hardware y software abierta para el desarrollo de proyectos electrónicos y de programación.}}
\newglossaryentry{ble}{name={Bluetooth Low Energy},description={Tecnología de comunicación inalámbrica diseñada para bajo consumo energético.}}
% \newglossaryentry{base64}{name={Base64},description={Codificación de texto para representar datos binarios, utilizada comúnmente para transmitir datos de manera segura.}}
\newglossaryentry{cmac}{name={Cipher-based Message Authentication Code},description={Método para asegurar la integridad y autenticidad de mensajes cifrados.}}

\newacronym{cmac-acronym}{CMAC}{Código de Autenticación de Mensaje basado en Cifrado}


\newglossaryentry{css}{name={Chirp Spread Spectrum},description={Técnica de modulación utilizada en LoRa para transmitir datos a largas distancias con baja potencia.}}

\newacronym{css-acronym}{CSS}{Espectro Ensanchado de Chirp}


\newglossaryentry{fhss}{name={Frequency-Hopping Spread Spectrum},description={Tecnología que cambia frecuentemente la frecuencia de transmisión para evitar interferencias.}}
\newglossaryentry{fsk}{name={Frequency Shift Keying},description={Método de modulación para transmitir datos digitales mediante cambios en la frecuencia de la señal portadora.}}
\newglossaryentry{freertos}{name={FreeRTOS},description={Sistema operativo en tiempo real utilizado en microcontroladores para gestionar múltiples tareas de manera eficiente.}}
\newglossaryentry{glonass}{name={Global Navigation Satellite System},description={Sistema ruso de navegación por satélite que complementa al GPS.}}
\newglossaryentry{gnss}{name={Global Navigation Satellite System},description={Término genérico para sistemas de posicionamiento global que incluyen GPS, GLONASS, entre otros.}}
\newglossaryentry{gps}{name={Global Positioning System},description={Sistema de navegación por satélite que proporciona datos de localización precisos.}}
\newglossaryentry{gsm}{name={Global System for Mobile Communications},description={Tecnología de comunicación móvil utilizada para voz y datos.}}
\newglossaryentry{iot}{name={Internet of Things},description={Red de dispositivos interconectados que recopilan y comparten datos en tiempo real para automatizar procesos.}}
\newglossaryentry{json}{name={JavaScript Object Notation},description={Formato ligero para el intercambio de datos, fácil de leer y escribir.}}
\newglossaryentry{lte}{name={Long Term Evolution},description={Tecnología de redes móviles de alta velocidad.}}
\newglossaryentry{lorawan}{name={Long Range Wide Area Network},description={Protocolo de red para dispositivos IoT que permite transmisiones de datos de largo alcance y bajo consumo.}}
\newglossaryentry{mac}{name={Media Access Control},description={Identificador único para dispositivos de red en comunicaciones.}}

\newacronym{mac-acronym}{MAC}{Control de Acceso al Medio}

\newglossaryentry{backhaul-glossary}{
    name={Backhaul},
    description={Segmento de la red que conecta los nodos de acceso (como estaciones base o gateways) a la red principal o al backbone, permitiendo la transmisión de datos de manera eficiente.}
}


\newglossaryentry{miso}{name={Master In Slave Out},description={Línea de datos utilizada en comunicación SPI.}}
\newglossaryentry{mosi}{name={Master Out Slave In},description={Línea de datos utilizada en comunicación SPI.}}
\newglossaryentry{mqtt}{name={Message Queuing Telemetry Transport},description={Protocolo ligero para la transmisión de mensajes entre dispositivos IoT.}}
\newglossaryentry{nfc}{name={Near Field Communication},description={Tecnología inalámbrica de corto alcance utilizada en pagos móviles y autenticación.}}
\newglossaryentry{otaa}{name={Over-The-Air Activation},description={Método de activación de dispositivos LoRaWAN para generar claves de sesión de forma dinámica.}}
\newglossaryentry{rfid}{name={Radio Frequency Identification},description={Tecnología de identificación y seguimiento mediante ondas de radio.}}
\newglossaryentry{spi}{name={Serial Peripheral Interface},description={Protocolo de comunicación serial para dispositivos embebidos.}}
\newglossaryentry{sx1262}{name={SX1262},description={Chip de comunicación LoRa utilizado en aplicaciones IoT.}}
\newglossaryentry{tdoa}{name={Time Difference of Arrival},description={Técnica para determinar la ubicación de un dispositivo mediante la diferencia de tiempo en la llegada de señales.}}
\newglossaryentry{thingsboard}{name={ThingsBoard},description={Plataforma de gestión de dispositivos IoT y visualización de datos en tiempo real.}}
\newglossaryentry{uart}{name={Universal Asynchronous Receiver-Transmitter},description={Protocolo para la comunicación serial asíncrona entre dispositivos.}}

\newacronym{rssi-acronym}{RSSI}{Indicador de Intensidad de Señal Recibida}

\newacronym{toa-acronym}{TOA}{Time on Air (Tiempo en el Aire)}
\newglossaryentry{toa-glossary}{
    name={Time on Air},
    description={Duración total que un mensaje ocupa en el medio de transmisión durante su envío, calculado en función del tamaño del mensaje, el factor de dispersión (SF), el ancho de banda y otros parámetros. Es un indicador clave en LoRaWAN.}
}
\newacronym{i2s-acronym}{I2S}{Inter-IC Sound (Interconexión de Circuitos Integrados de Sonido)}

\newacronym{i2c-acronym}{I2C}{Inter-Integrated Circuit (Interconexión de Circuitos Integrados)}

\newacronym{lns-acronym}{LNS}{LoRaWAN Network Server (Servidor de Red LoRaWAN)}
\newglossaryentry{lns-glossary}{
    name={LoRaWAN Network Server},
    description={Componente central en una red \gls{lorawan-glossary} que gestiona la autenticación, la enrutación de datos, y el control de dispositivos finales. Coordina las comunicaciones entre gateways y aplicaciones.}
}



% Acrónimos
\newacronym{loraw}{LoRaWAN}{Red de Área Amplia de Largo Alcance}
\newacronym{rf}{RF}{Radiofrecuencia}
\newacronym{iot-acronym}{IoT}{Internet de las Cosas}
\newacronym{rfid-acronym}{RFID}{Identificación por Radiofrecuencia}
\newacronym{gps-acronym}{GPS}{Sistema de Posicionamiento Global}
\newacronym{gnss-acronym}{GNSS}{Sistema Global de Navegación por Satélite}
\newacronym{lora-acronym}{LoRa}{Largo Alcance}
\newacronym{uart-acronym}{UART}{Universal Asynchronous Receiver-Transmitter (Receptor-Transmisor Asíncrono Universal)}
\newacronym{spi-acronym}{SPI}{Serial Peripheral Interface (Interfaz Periférica Serial)}
\newacronym{api-acronym}{API}{Application Programming Interface (Interfaz de Programación de Aplicaciones)}
\newacronym{json-acronym}{JSON}{JavaScript Object Notation (Notación de Objetos de JavaScript)}
\newacronym{gsm-acronym}{GSM}{Global System for Mobile Communications (Sistema Global para Comunicaciones Móviles)}

\newacronym{sms-acronym}{SMS}{Short Message Service (Servicio de Mensajes Cortos)}




\newacronym{otaa-acronym}{OTAA}{Activación por el Aire}
\newacronym{ble-acronym}{BLE}{Bluetooth Low Energy (Bluetooth de Baja Energía)}
\newacronym{fhss-acronym}{FHSS}{Frequency-Hopping Spread Spectrum (Espectro Ensanchado por Salto de Frecuencia)}
\newacronym{nfc-acronym}{NFC}{Near Field Communication (Comunicación de Campo Cercano)}
\newacronym{lte-acronym}{LTE}{Long Term Evolution (Evolución a Largo Plazo)}

% Glosario
\newglossaryentry{iot-glossary}{name={Internet of Things},description={Red de dispositivos interconectados que recopilan y comparten datos en tiempo real para automatizar procesos.}}
\newglossaryentry{rfid-glossary}{name={Radio Frequency Identification},description={Tecnología de identificación y seguimiento mediante ondas de radio.}}
\newglossaryentry{gps-glossary}{name={Global Positioning System},description={Sistema de navegación por satélite que proporciona datos de localización precisos.}}
\newglossaryentry{gnss-glossary}{name={Global Navigation Satellite System},description={Término genérico para sistemas de posicionamiento global que incluyen GPS, GLONASS, entre otros.}}
\newglossaryentry{lorawan-glossary}{name={Long Range Wide Area Network},description={Protocolo de red para dispositivos IoT que permite transmisiones de datos de largo alcance y bajo consumo.}}
\newglossaryentry{uart-glossary}{name={Universal Asynchronous Receiver-Transmitter},description={Protocolo para la comunicación serial asíncrona entre dispositivos.}}
\newglossaryentry{spi-glossary}{name={Serial Peripheral Interface},description={Protocolo de comunicación serial para dispositivos embebidos.}}
\newglossaryentry{api-glossary}{name={Application Programming Interface},description={Conjunto de herramientas y definiciones que permiten la interacción entre aplicaciones de software.}}
\newglossaryentry{json-glossary}{name={JavaScript Object Notation},description={Formato ligero para el intercambio de datos, fácil de leer y escribir.}}
\newglossaryentry{gsm-glossary}{name={Global System for Mobile Communications},description={Tecnología de comunicación móvil utilizada para voz y datos.}}

\newglossaryentry{otaa-glossary}{name={Over-The-Air Activation},description={Método de activación de dispositivos LoRaWAN para generar claves de sesión de forma dinámica.}}
\newglossaryentry{ble-glossary}{name={Bluetooth Low Energy},description={Tecnología de comunicación inalámbrica diseñada para bajo consumo energético.}}
\newglossaryentry{fhss-glossary}{name={Frequency-Hopping Spread Spectrum},description={Tecnología que cambia frecuentemente la frecuencia de transmisión para evitar interferencias.}}
\newglossaryentry{nfc-glossary}{name={Near Field Communication},description={Tecnología inalámbrica de corto alcance utilizada en pagos móviles y autenticación.}}
\newglossaryentry{lte-glossary}{name={Long Term Evolution},description={Tecnología de redes móviles de alta velocidad.}}

